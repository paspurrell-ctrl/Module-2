\documentclass{article}
\usepackage{graphicx} % Required for inserting images
\usepackage{float}
\usepackage[margin=1in]{geometry}
\usepackage{url}
\usepackage{xurl}
\title{Math 2030 module 2}
\author{Paige Spurrell}
\date{February 2026}

\begin{document}
\pagenumbering{roman}
\maketitle
\newpage
\tableofcontents
\maketitle
\newpage
\section{Introduction}
At some point in most science courses, you will come across population growth models, and to understand them and their importance, it all starts with the calculus behind each equation. In applied mathematics, the change of quantities over time, like population, is represented by differential equations. Population growth models are extremely important in different branches of biology. For example, expressing the rate of change in animal populations can help with more in-depth research done on how those animals evolve under different environmental conditions and resource availability (Khan Academy).
There have been multiple models proposed to describe the rate of change. Some big examples are the linear growth model, which has a constant rate of increase over time, and the exponential growth model, which assumes the current population size is proportional to the rate of growth. There is also the logistic population growth model, which has an extra added component. This model incorporates the carrying capacity, which is the maximum population size an environment can support. This project will explore linear, exponential, and logistic growth models and aims to clarify when each model is used and when its assumptions become unrealistic.
\section{Methods}
To understand each model, we start by looking at the general equation for all the types of models. The population growth rate is the change in numbers of individuals in a population over time (Khan Academy). 
\begin{equation}
    \centering
     \frac{\mathrm{d}N}{\mathrm{d}t} = rN
\end{equation}
This equation is the basis for linear, exponential, and logistic growth models. You rarely use this one due to the lack of components that the equation requires.
In the  equation, dN/dT is the rate of growth the population is experiencing. N is the population size, T is time and r is the per capita rate of increase (number of deaths subtracted from the number of births).

\subsection{Exponential Growth Model}
Exponential growth is used when a population increases faster the more time passes (Chen, 2025). This creates a J-shaped curve on a graph. Many different populations can be represented using an exponential growth model, like a rapidly growing population of animals, which we often see in small rodents, or in the growth of bacteria cultured in laboratory conditions. 

\begin{equation}
    \centering
     \frac{\mathrm{d}N}{\mathrm{d}t} = r_{\max} N
\end{equation}
This differential equation describes the rate at which the population changes at any given moment. It explains how the population is changing at any given moment but to use this model there also need to be a lot of assumptions made. 

Exponential growth assumes there are unlimited resources, no environmental constraints, and a constant per capita growth rate as well as no immigration or emigration. Although these assumptions make the math easier, they reduce the realism in any long-term results. The small difference between the general equation and this exponential one is the r{max}, which represents the maximum possible growth rate a population can achieve when resources are unlimited. 
\begin{equation}
N(t) = N_0 e^{r_{\max} t}
\end{equation}
This is the explicit solution that shows that population size grows proportionally to an exponential function of time. This is how you would find the actual population size at time t.
\begin{figure}[H]

    \centering
    \includegraphics[width=0.4\linewidth]{exponential growth.png}
    \caption{Exponential Growth Graph}

\end{figure}
As shown in graph 1, there is accelerating growth leading to a strong J-shaped curve in the population trajectory. This is due to the growth rate being proportional to the current population size. We can also observe from this graph that long-term behavior, or as t goes to infinity, the exponential solution grows without bound, which is very unrealistic in nature.

Exponential growth models do have their limitations, but they are useful when analyzing fast-growing animal populations, the spread of diseases, and growing bacteria (Chen, 2025). It's great for providing reasonably accurate approximations over short time intervals or in environments where resources are only temporarily abundant. 
\subsection{Linear Growth Model}
A linear exponential growth model ensures what exponential cannot; it offers predictability and stability (Statsig, 2024). Unlike exponential growth where the rate of increase depends on the population size, linear growth assumes that the population grows by the same amount in each unit of time (Libretexts).
\begin{equation}
\frac{\mathrm{d}N}{\mathrm{d}t} = c
\end{equation}
The differential equation for this growth model is very straight forward. it simpily states that the rate of change of the population is constant, where c is a fixed growth rate. Solving the differential equations gives us the explicit solution:
\begin{equation}
N(t) = N_0 + ct,
\end{equation}
In this equation the N$x_0$ represents the initial population size and it can be used find the population size at any time, t. 

Linear growths increase in population and remain steady and easily predicted, unlike exponential growth that accelerates over time. This model is most useful for short-term projections and initial growth analysis due to its strong assumptions (Pearson). It assumes that the population increases  by the same unit of time regardless of how large the population is. This model also says that outside conditions affecting growth remain constant. This can only be accurate under controlled laboratory experiments, not in the wild like animal populations.
\begin{figure}[H]
    \centering
    \includegraphics[width=0.4\linewidth]{linear.png}
    \caption{Linear Population Growth}
\end{figure}
As shown in figure 2, similarly to exponential growth, as t goes to infinity the linear solution grows without bound when c is greater than 0. Since the model does not take into account all the environmental limitations, it's most appropriate in short-term approximations.
\subsection{Logistic Growth Model}

\section{Results}
\end{document}
