\documentclass{article}
\usepackage{graphicx} % Required for inserting images
\usepackage{float}
\usepackage[margin=1in]{geometry}
\usepackage{url}
\usepackage{xurl}
\title{Math 2030 module 2}
\author{Paige Spurrell}
\date{February 2026}

\begin{document}
\pagenumbering{roman}
\maketitle
\newpage
\tableofcontents
\maketitle
\newpage
\section{Introduction}
At some point in most science courses you will come across population growth models and to understand them and their importance, it all starts with the calculus behind each equation. In applied mathematics, the change of quantities over time, like population, is represented by differential equations. Population growth models are extremely important in different branches of biology. By expressing the rate of change in animal populations, there can be more in-depth research done on how those animals evolve under different environmental conditions and resource availability.
There have been multiple models proposed to describe rate of change. Some big examples is the linear growth model which has a constant rate of increase over time and the exponential growth model which assumes the current population size is proportional to the rate of growth. There is also the logistic population growth model which has an extra added component. This model incorporates the carrying capacity, which is the maximum population size an environment can support. This project will explore linear, exponential and logistic growth models and aims to clarify when each model is used and when its assumptions become unrealistic.
\section{Methods}
\subsection{Linear Growth Model}
\subsection{Exponential Growth Model}
\subsection{Logistic Growth Model}
\end{document}
